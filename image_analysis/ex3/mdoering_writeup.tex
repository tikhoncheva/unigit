\documentclass[11pt]{scrartcl}

%\usepackage[ngerman]{babel}
\usepackage[utf8]{inputenc}
\usepackage{graphicx}

\usepackage{amsmath}
\usepackage{amsthm}
\usepackage{mathtools}

\usepackage{listings}

%%%%% COMMANDS

\newcommand{\FT}{\mathcal{F}}
\newcommand{\T}{\mathrm{T}}
\newcommand{\IFT}{\mathcal{F}^{-1}}
\newcommand{\conv}{\ast}
\newcommand{\defined}{\coloneqq}

\newtheorem*{theorem}{Theorem}



\begin{document}



\title{Image Analysis Excercise Sheet 3}
\author{Markus Doering, 3153320}
\maketitle

\section{Convolution in 1-D}
Given are an input signal $g$ and a filter $h$, which will act on the signal: 
%
\begin{align*}
g(x) &= \begin{pmatrix}0  & 0  & 0  & 0  & 1  & 1  & 1  & 0  & 0  & 0 \end{pmatrix}^\T \\
h(x) &= \begin{pmatrix}0  & 0  & 0  & 0  & 1  & 2  & 1  & 0  & 0  & 0 \end{pmatrix}^\T. 
\end{align*}
%
The output signal will be 
%
\begin{equation*}
(g\conv h)(x) = \begin{pmatrix}4  & 3  & 1  & 0  & 0  & 0  & 0  & 0  & 1  & 3 \end{pmatrix}^\T,
\end{equation*}
%
a smoothened, shifted and amplified version of the signal. The circulant matrix of the filter is given by 
%
\begin{equation*}
H = %
 \begin{pmatrix}
  0  & 0  & 0  & 0  & 1  & 2  & 1  & 0  & 0  & 0  \\
  0  & 0  & 0  & 0  & 0  & 1  & 2  & 1  & 0  & 0  \\
  0  & 0  & 0  & 0  & 0  & 0  & 1  & 2  & 1  & 0  \\
  0  & 0  & 0  & 0  & 0  & 0  & 0  & 1  & 2  & 1  \\
  1  & 0  & 0  & 0  & 0  & 0  & 0  & 0  & 1  & 2  \\
  2  & 1  & 0  & 0  & 0  & 0  & 0  & 0  & 0  & 1  \\
  1  & 2  & 1  & 0  & 0  & 0  & 0  & 0  & 0  & 0  \\ 
  0  & 1  & 2  & 1  & 0  & 0  & 0  & 0  & 0  & 0  \\
  0  & 0  & 1  & 2  & 1  & 0  & 0  & 0  & 0  & 0  \\
  0  & 0  & 0  & 1  & 2  & 1  & 0  & 0  & 0  & 0
 \end{pmatrix}.
\end{equation*}
%
% The convolution theorem tells us that 
% %
% \begin{equation*}
% (g\conv h)(x) = \IFT\left(\FT g\odot\FT h\right)(x),
% \end{equation*}
% %
% and therefore 
%
The convolution of $g$ and $h$ can be computed as 
%
\begin{equation*}
 g\conv h = H\cdot g.
\end{equation*}
%
\begin{theorem}
Let $g$ be a signal and let $h$ be a filter with 
\[
\sum_k (h\conv g)(k) = \sum_k g(k). 
\]
In the general case $\sum_k g(k) \neq 0$, we find that the DC term $\sum_k h(k) = 1$.
\end{theorem}
\begin{proof}
Define $s\defined(1,\ldots,1)^\T$.
%
\begin{align*}
\sum_j g(j) &= \sum_k (h\conv g)(j) = s^\T(h\conv g) \stackrel{\text{\tiny Convolution Theorem}}{=}\\
            &= s^\T \IFT\left(\FT f\odot\FT g\right) = s^\T F^\dagger\left(Ff\odot Fg\right)\frac{1}{n} = \\
            &= (1,0,0,\ldots,0)\left(Ff\odot Fg\right) = \left(s^\T f\cdot s^\T g\right) = \\
            &= \left(\sum_k h(k)\right)\left(\sum_j g(j)\right)
\end{align*}
%
We can divide both sides by the signal's sum and get $\sum_k h(k) = 1$.
\end{proof}

\section{DFT in 1-D}
Given is the transformed function 
\[
 \mathring g(k) = \sqrt{n}\frac{i}{2}\left(\delta_{+2}(k) - \delta_{-2}(k)\right),
\]
and we want to compute the original function by discrete Fourier transform. Let $F_{l,\cdot}$ denote the $l$-th row of the Fourier matrix.
\begin{align*}
g(l) &= \frac{1}{\sqrt n} F_{l,\cdot} \mathring g= \frac i 2 \sum_{k=0}^{n-1}e^{\frac{2\pi i}{n} l k}\left(\delta_{+2}(k) - \delta_{-2}(k)\right) \\
     &= \frac i 2 \left(e^{\frac{4\pi i}{n} l} - e^{-\frac{4\pi i}{n}l}\right) = \frac i 2 \left(2i\sin\left(\frac{4\pi}{n}l\right)\right) = -\sin\left(\frac{4\pi}{n}l\right)
\end{align*}

\section{DFT in 2-D}
When two matrices have no spectral overlap, the convolution theorem asserts that 
\[
 a\conv b = \IFT\left(\FT(a)\odot\FT(b)\right) = \IFT(0) = 0.
\]
The python code is found in the appendix.


\section{Correlation Theorem}

\begin{theorem} Cross Correlation Theorem
\[
 \FT\left(a\star g\right) = \FT(a)^*\odot\FT(g)
\]
\end{theorem}
\begin{proof}
We observe that the cross correlation operation can be stated in terms of the circulant matrix $A$ of $a$ as
\[
 a\star g = A^\dagger g.
\]
We recall the equation 
\[
 AW = W\mathring A {\ }\Leftrightarrow{\ } W^\dagger A^\dagger = \mathring{A}^\dagger W^\dagger {\ }\Leftrightarrow{\ } W^\dagger A^\dagger W = \mathring{A}^*,
\]
where $\mathring A$ is the diagonal matrix with entries $\mathring{a} = \FT a$ and $W$ is the normalized Fourier matrix ($WW^\dagger=I$).
We obtain
\begin{align*}
\FT\left(a\star g\right) &= \FT\left(A^\dagger g\right) = W^\dagger A^\dagger g = W^\dagger A^\dagger W W^\dagger g\\
			 &= \mathring{A}^* W^\dagger g = \FT(a)^*\odot\FT(g).
\end{align*}
\end{proof}

\newpage
\appendix
\lstinputlisting[language=python]{ex3.py}
\end{document}