\documentclass[11pt]{scrartcl}

%\usepackage[ngerman]{babel}
\usepackage[utf8]{inputenc}
\usepackage{graphicx}

\usepackage{amsmath}
\usepackage{amsthm}

%%%%% COMMANDS

\newcommand{\FT}{\mathcal{F}}
\newcommand{\IFT}{\mathcal{F}^{-1}}
\newcommand{\conv}{\ast}

\newtheorem*{theorem}{Theorem}



\begin{document}



\title{Image Analysis Excercise Sheet 3}
\author{Markus Doering, 3153320}
\maketitle

\section{Convolution in 1-D}
Given are an input signal $g$ and a filter $h$, which will act on the signal: 
%
\begin{align*}
g(x) &= \begin{pmatrix}0  & 0  & 0  & 0  & 1  & 1  & 1  & 0  & 0  & 0 \end{pmatrix}^\mathrm T \\
h(x) &= \begin{pmatrix}0  & 0  & 0  & 0  & 1  & 2  & 1  & 0  & 0  & 0 \end{pmatrix}^\mathrm T. 
\end{align*}
%
The output signal will be 
%
\begin{equation*}
(g\conv h)(x) = \begin{pmatrix}4  & 3  & 1  & 0  & 0  & 0  & 0  & 0  & 1  & 3 \end{pmatrix}^\mathrm T,
\end{equation*}
%
a smoothened, shifted and amplified version of the signal. The circulant matrix of the filter is given by 
%
\begin{equation*}
H = %
 \begin{pmatrix}
  0  & 0  & 0  & 0  & 1  & 2  & 1  & 0  & 0  & 0  \\
  0  & 0  & 0  & 0  & 0  & 1  & 2  & 1  & 0  & 0  \\
  0  & 0  & 0  & 0  & 0  & 0  & 1  & 2  & 1  & 0  \\
  0  & 0  & 0  & 0  & 0  & 0  & 0  & 1  & 2  & 1  \\
  1  & 0  & 0  & 0  & 0  & 0  & 0  & 0  & 1  & 2  \\
  2  & 1  & 0  & 0  & 0  & 0  & 0  & 0  & 0  & 1  \\
  1  & 2  & 1  & 0  & 0  & 0  & 0  & 0  & 0  & 0  \\ 
  0  & 1  & 2  & 1  & 0  & 0  & 0  & 0  & 0  & 0  \\
  0  & 0  & 1  & 2  & 1  & 0  & 0  & 0  & 0  & 0  \\
  0  & 0  & 0  & 1  & 2  & 1  & 0  & 0  & 0  & 0
 \end{pmatrix}.
\end{equation*}
%
% The convolution theorem tells us that 
% %
% \begin{equation*}
% (g\conv h)(x) = \IFT\left(\FT g\odot\FT h\right)(x),
% \end{equation*}
% %
% and therefore 
%
The convolution of $g$ and $h$ can be computed as 
%
\begin{equation*}
 g\conv h = H\cdot g.
\end{equation*}
%
\begin{theorem}
Let $g$ be a signal and let $h$ be a filter with 
\[
\sum_k (h\conv g)(k) = \sum_k g(k). 
\]
In the general case $\sum_k g(k) \neq 0$, we find that $\sum_k h(k) = 1$.
\end{theorem}
\begin{proof}
 Left to the reader as an exercise.
\end{proof}




\end{document}