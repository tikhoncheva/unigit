\documentclass[a4paper,11pt, BCOR=4mm, DIV=12, pagesize]{scrartcl}


\thispagestyle{empty}
\pagestyle{empty}
\usepackage[utf8]{inputenc}
%\usepackage{ngerman}
%\usepackage[ngerman]{babel}
\usepackage[babel,german=quotes]{csquotes}
%\usepackage{exscale}
%\usepackage{dsfont}
%\usepackage{latexsym}
%\usepackage{theorem}   % change theoremstyle
%\usepackage[dvips]{color}


\usepackage{graphicx}
\usepackage{subfigure} 
\usepackage[singlelinecheck=true]{caption} % working out of ~
\usepackage{color}
% \usepackage{booktabs}
\usepackage{enumerate}
\usepackage{textcomp}
\usepackage{xparse}



\usepackage{amsmath}
\allowdisplaybreaks[1]
\usepackage{colonequals}
\usepackage{mathtools}
\usepackage{amsthm}
\usepackage{amssymb}
\usepackage{amsfonts}
\usepackage{bbm}
\usepackage{braket}


\usepackage[plainpages=false]{hyperref}
\definecolor{darkblue}{rgb}{0,0,.5}
\hypersetup{colorlinks=true, breaklinks=true, linkcolor=darkblue, 
menucolor=darkblue, urlcolor=darkblue}

\usepackage{setspace}


%==============================================================================
% theorem stuff
%\theoremstyle{break}  % add line break after theorem header
\newtheorem{thm}{Theorem}[section]
\newtheorem{cor}[thm]{Corollary}
\theoremstyle{definition}
\newtheorem{definition}[thm]{Definition}

% \newtheorem{satz}{Satz}[section]
% \newtheorem{lemma}[satz]{Lemma}
% \newtheorem{korollar}[satz]{Korollar}
% 
% \newtheorem{bsp}[satz]{Beispiel}
% \newtheorem{bem}[satz]{Bemerkung}
% \newtheorem{folgerung}[satz]{Folgerung}

% \newcommand*\defnpenalties{%
%   % Geklaut aus einem Posting von Michael Downes:
%   % > Set \clubpenalty, etc. to distinctive values for use
%   % > in tracing page breaks. These values are chosen so that
%   % > no single penalty will absolutely prohibit a page break,
%   % > but certain combinations of two or more will.
%   %
%   \clubpenalty=9996 \widowpenalty=9999 \brokenpenalty=4991
%   %
%   % > Reiterate the default value of \predisplaypenalty, for
%   % > completeness.
%   % >
%   % > Set postdisplaypenalty to a fairly high value to discourage
%   % > a page break between a display and a widow line at the end
%   % > of a paragraph.
%   %
%   \predisplaypenalty=10000 \postdisplaypenalty=1549
%   %
%   % > And then \displaywidowpenalty should be at least as high as
%   % > \postdisplaypenalty, otherwise in a situation where two
%   % > displays are separated by two lines, TeX will prefer to
%   % > break between the two lines, rather than before the first
%   % > line.
%   %
%   \displaywidowpenalty=1602
% }%
% 
% \let\sdef\definition
% \renewcommand{\def}{\defnpenalties\sdef}
% \let\sthm\thm
% \renewcommand{\thm}{\defnpenalties\sthm}
% \let\scor\cor
% \renewcommand{\cor}{\defnpenalties\scor}

%==============================================================================
% counter stuff
\renewcommand{\thefigure}{\thesection.\arabic{figure}}
\renewcommand{\theequation}{\thesection.\arabic{equation}}
\renewcommand{\thetable}{\thesection.\arabic{table}}

\newcommand{\resetsectioncounters}{%
\setcounter{figure}{0}%
\setcounter{table}{0}%
\setcounter{equation}{0}%
}




%==============================================================================
% my commands

\newcommand{\R}{\mathbb{R}} % real numbers
\newcommand{\N}{\mathbb{N}} % natural numbers
\newcommand{\C}{\mathbb{C}} % complex numbers
\renewcommand{\P}{\mathbb{P}}
\newcommand{\ind}{\mathbbm{1}}
\newcommand{\E}{\mathbb{E}}
\NewDocumentCommand\cmat{mg}{%
    \ensuremath{\C^{#1\times \IfNoValueTF{#2}{#1}{#2}}}%
}
\newcommand{\ugroup}[1]{U(#1)}
\newcommand{\urep}[1]{U_{#1}}


\newcommand{\needcite}{\textcolor{red}{[citation needed]}}
\newcommand{\needref}{\textcolor{red}{[local reference needed]}}
\newcommand{\needproof}{\textcolor{red}{[proof needed]}}
\newcommand{\todo}{\textcolor{red}{\Large[TODO]}}

\newcommand{\eps}{\varepsilon}
\renewcommand{\phi}{\varphi}
\newcommand{\Id}{\ \mathrm{d}}

\newcommand{\figref}[1]{(\ref{#1}, p. \pageref{#1})}


%==============================================================================
% my environments

\newenvironment{beweis}{\begin{proof}[Beweis]\defnpenalties}{\end{proof}}

%==============================================================================
%global settings
%\pagestyle{headings}

%\usepackage{a4wide}
%\usepackage{fullpage}
\pagestyle{plain}


\setcounter{tocdepth}{2}

\hyphenation{Zu-falls-stich-pro-be Zu-falls-stich-pro-ben Kon-fi-denz-wert Kon-fi-denz-wer-ten Kon-fi-denz-in-ter-vall}


%==============================================================================
% my settings






\onehalfspacing
\KOMAoptions{DIV=last}

\begin{document}
    \title{Random Unitary Crap}
\date{November 12, 2013}
\author{Markus D\"oring\\University of Heidelberg}
\maketitle
%\cleardoublepage
% \clearpage
% \pagenumbering{Roman}

	%\addcontentsline{toc}{chapter}{Inhalt}
% \begin{spacing}{1}
%     \tableofcontents
% 	\addcontentsline{toc}{section}{Inhaltsverzeichnis}
% 	\cleardoublepage  
%    \listoffigures
% 	\addcontentsline{toc}{section}{Abbildungsverzeichnis}
% 	\cleardoublepage  
% %     \listoftables
% % 	\addcontentsline{toc}{section}{Tabellenverzeichnis}
% % 	\cleardoublepage  
% \end{spacing}


\pagenumbering{arabic}

%%%%%%%%%%%%%%%%%%%%%%%%%%%%%%%%%%%%%%%%%%%%%%%%%%%%%%%%%%%%%%%%%%%%%%%%%%%%%%%%
% ACTUAL SEMINAR STUFF
\section{Preface}

This is the written report corresponding to my talk in the seminar 
\emph{Selected topics in Mathematical Physics: Quantum information theory}. The 
seminar was lead by Prof. Dr. Manfred Salmhofer, executive director of the 
Institute for Theoretical Physics at the University of Heidelberg, and Dr. 
Markus M\"uller, junior research group leader at that institute. 

The talk starts with a review of some properties of pure quantum states, 
unitary matrices and density matrices, followed by the introduction of group 
representations and the Haar measure. The final result will be the proof of 
Schur's Lemma and its use for calculating expectation values of random pure 
states.

\section{Motivation}

In fields like statistical dynamics or chaos theory, we use probabilistic 
models to come to conclusions about the state and the evolution of a system. 
If we want to apply quantum theory to these fields, it would be helpful to 
know whether a state whose classical analog is chaotic is effectively random 
\cite{wootters}. 

For the definition of a \emph{random quantum state} we need at least a 
probability measure on the set of quantum states. It turns out that the 
\emph{Haar measure} can be seen as a probability measure, and that 
some results of group representation theory allow us to calculate some results 
for random quantum states. 

In this seminar talk, we will start with some basics on the unitary group, then 
we will introduce the Haar measure as a probability measure on the unitary 
group, and we will close with the calculation of expectation values of random 
pure states:
\begin{equation}
\E(\ket{\psi}) = \int_{\ugroup{n}} \ket{\psi}\Id |\psi\rangle.
\end{equation}

\section{The Unitary Group}\label{sec:ugroup}
Basics... \todo

\section{Unitary Group Representations}
In this section we will introduce the notion of \emph{group representation} and 
examine its implications for the statistical properties that we want to 
analyze. For a more detailed view on the subject, refer to \cite{rep}. Most of 
the upcoming definitions are analog to the ones given there, but simplified to 
our use case where possible. 

Lets start with some basic notation conventions and definitions.

\begin{notation}
 We will use the following naming conventions throughout the next section. Let 
$V$ be a complex vector space of dimension $0<n<\infty$. Denote the set of maps 
from $V$ to itself as $\Endos{V}$ (endomorphisms on $V$) and the invertible 
endomorphisms on $V$ as $\Autos{V}$ (automorphisms on $V$).
\end{notation}

\begin{definition}(groups and group homomorphisms)\label{def:group}\\
 A group $(G,\cdot)$ is a set $G$ combined with a multiplication operator that 
fulfills the following axioms: 
\begin{description}
 \item[Closure] $\forall g,h\in G:\ g\cdot h\in G$
 \item[Associativity] $\forall g,h,i\in G:\ (g\cdot h)\cdot i = g\cdot (h\cdot 
i)$
 \item[Existence of Identity] $\exists e\in G\ \forall g\in G:\ g\cdot e = 
e\cdot g = g$
 \item[Existence of Inverse] $\forall g\in G\ \exists g^{-1}\in G:\ g\cdot 
g^{-1} = e$
\end{description}
A group homomorphism is a map between two groups $(G,\cdot), (H,\circ)$ that 
preserves the group structure, i.e.
\begin{align*}
 f:\quad G&\to H\\
 g &\mapsto f(g)\\
 \forall u,v\in G:\quad f(u\cdot v) &= f(u)\circ f(v)\\
 f(u^{-1}) &= f(u)^{-1}
\end{align*}

\end{definition}

\begin{example}
 The unitary group $\ugroup{n}$ is a group with identity element $\ind$ and 
inverse elements $U^{-1} = U^\dagger$.
\end{example}

\begin{definition}(group representation, $G$-module)\label{def:gmod}\\
 A group homomorphism
 \begin{align}\begin{split}
  \urep{(\cdot)}:\quad G&\to \Autos{V}\\
  g&\mapsto \urep{g}
  \end{split}
 \end{align}
 is called a \emph{group representation} on $V$, and $V$ is then referred to as 
a $G$-module. If $\urep{g}\in\ugroup{n}\ \forall g\in G$, we call $U$ a 
\emph{unitary group representation}. If $\urep{(\cdot)}$ is bijective, the 
representation is called \emph{faithful}.
\end{definition}

We want to use only unitary representations in this section, but most results 
also apply to general representations as well, because every group 
representation can be seen as a unitary representation with respect to some 
specific inner product (see \cite{rep}, p. 68f, for details).

\begin{definition}($G$-submodule, irreducibility)\label{def:irred}\\
 Let $\urep{g}$ be a unitary group representation of group $G$ on $V$. A 
subspace $V_1\subseteq V$ is called invariant (under the group action), if 
\begin{equation}
 \forall g\in G, \ket{v}\in V_1:\ U_g\ket{v}\in V_1.
\end{equation}
An invariant subspace $V_1$ of $V$ is also called a $G$-submodule of $V$.

If the only $G$-submodules of $V$ are $V$ and $\{0\}$, we call $V$ an 
irreducible $G$-module and $\urep{g}$ an \emph{irreducible representation}. 
Otherwise, we call both \emph{reducible}. $G$-submodules that are not trivial 
(i.e. $\{0\}$ or $V$) are called \emph{proper $G$-submodules}. 
\end{definition}

\begin{example}\ \\
 Consider the unitary representation 
 \begin{align*}
  \urep{(\cdot)}: \ugroup{n} &\to \Autos{\C^n\otimes\C^n}\\
  U&\mapsto U\otimes U.
 \end{align*}
 It is a reducible representation with the irreducible invariant subspaces 
 \begin{align*}
  \C^n_{\text{sym}} &\coloneqq \left\{\ket{\psi}\in\C^n\otimes\C^n:\  
\F(\ket{\psi}) = \ket{\psi}\right\}\\
  \C^n_{\text{antisym}} &\coloneqq \left\{\ket{\psi}\in\C^n\otimes\C^n: 
\F(\ket{\psi}) = -\ket{\psi}\right\},
 \end{align*}
 where $\F$ denotes the flip operator:
 \begin{equation}
  \F(\ket{\phi}\otimes\ket{\psi}) = \ket{\psi}\otimes \ket{\phi}
 \end{equation}

\end{example}
\begin{proof}
\todo Example Flip-Operator 
\end{proof}

\begin{prop}\ \label{thm:directsum}\\
 Let $\urep{g}$ be a group representation of the group $G$ on the $G$-module 
$V$.  
\begin{enumerate}\renewcommand{\labelenumi}{(\alph{enumi})}
 \item If $V_1\subset V$ is a $G$-submodule, then there is a $G$-submodule 
  $V_2\perp V_1$ with $V = V_1 \oplus V_2$. 
\item The $G$-module $V$ is the direct sum of irreducible, pairwise orthogonal 
  $G$-submodules.
  \item The decomposition of $V$ into orthogonal irreducible $G$-submodules is 
unique.
\end{enumerate}
\end{prop}
\begin{proof}
 Let $V_1$ be a $G$-submodule of $V$. If $V_1 = V$ or $V_1 = \{0\}$, the 
proposition (a) is trivial. Let therefore $m\in\{1,\ldots,n-1\}$ be the 
dimension of $V_1$. We can choose an orthonormal base $\ket{u_i}, 1\leq i\leq m$ 
of $V_1$ and extend it to an orthonormal base $\ket{u_i}, 1\leq i\leq n$ of $V$. 
Now we observe the group action on these base vectors. For an arbitrary, but 
fixed $g\in G$, we write 
\begin{equation}
 \urep{g} = \summe{i}{1}{n}\summe{j}{1}{n} a_{ij} \ket{u_i}\bra{u_j}.
\end{equation}
The group action on $\ket{u_k}, 1\leq i\leq m$ is therefore
\begin{equation}
 \urep{g}\ket{u_k} = \summe{i}{1}{n}\summe{j}{1}{n} a_{ij} 
\ket{u_i}\braket{u_j|u_k} = \summe{i}{1}{n} a_{ik}\ket{u_i}.
\end{equation}
This vector must lie in $\mathrm{span}(\{\ket{u_k}, 1\leq k\leq m\})$, and thus 
$\forall 1\leq k\leq m, m<i\leq n: a_{ik} = 0$. We can apply the same argument 
to the representation $\urep{g^{-1}} = \urep{g}^\dagger$, and find that 
$\forall 1\leq k\leq m, m<i\leq n: a_{ki} = 0$. We can now write 
\begin{equation}\label{eq:repsum}
 \urep{g} = \left(\summe{i}{1}{m}\summe{j}{1}{m} a_{ij} 
\ket{u_i}\bra{u_j}\right) + \left(\summe{i}{m+1}{n}\summe{j}{m+1}{n} a_{ij} 
\ket{u_i}\bra{u_j}\right),
\end{equation}
which shows us that $V_2$ is also invariant under the group action, and 
therefore a $G$-submodule of $V$.

The second part of the proposition is easily shown by induction: apply 
the arguments given here on the $G$-submodules $V_1$ and $V_2$ recursively.

The uniqueness is also a simple conclusion. Assume there are two distinct sets 
of irreducible $G$-submodules $V_1,\ldots,V_k$ and $W_1,\ldots,W_m$. Then 
(without loss of generality, just reordered) $V_1\cap W_1\supsetneq\{0\}$, 
$V_1\neq W_1$ and $W_1\supsetneq V_1\cap W_1$. Let $\ket{v}\in V_1\cap 
W_1\setminus\{0\}$, then by definition \ref{def:irred} $U_g\ket{v}\in V_1$ and 
$U_g\ket{v}\in W_1$ and thus $U_g\ket{v}\in V_1\cap W_1$. This means that 
$V_1\cap W_1$ is a proper $G$-submodule of $W_1$, which is a contradiction.
\end{proof}

This proposition also provides us with some insight about the structure of 
reducible group representations. As we can see in equation \ref{eq:repsum}, all 
the representation matrices are in fact block diagonal matrices with respect to 
the chosen orthonormal base, and therefore to any orthonormal base constructed 
from bases of the irreducible submodules.

The following result, Schur's Lemma, is a key ingredient for calculating 
statistical properties of random states, like the expectation value. In most of 
the literature  available, Schur's lemma is stated in terms of 
morphisms between $G$-modules, but we will use a less general version in 
terms of matrices, because it is sufficient for the calculation of the 
properties that we are interested in. 

\begin{lemma}(Issai Schur 1905)\label{lemma:schur}\\
Let $\urep{g}$ be a unitary 
 representation of the group $G$ on the complex vector space $V$. If a 
 diagonalizable matrix $A$ commutes with  $\urep{g}$ 
 for all $g\in G$, then A can be written as a linear combination of the 
projectors onto the invariant subspaces of $V$, i.e. 
 \begin{equation}
  A = \summe{i}{1}{m} \lambda_i \pi_i,
 \end{equation}
 with $\lambda_i\in\C$ and 
 \begin{equation}
  \pi_i \coloneqq \sum\limits_{j\in I_i} \ket{u_j}\bra{u_j}
 \end{equation}
 for an orthonormal base $\left\{\ket{u_j}:\ j\in I_i\subseteq 
\left\{1,\ldots,n\right\}\right\}$ of $V_i$ as in proposition 
\ref{thm:directsum}.
\end{lemma}
\begin{proof}
 We can state the commutativity of $A$ as 
 \begin{equation*}
  A\urep{g} = \urep{g}A\quad \forall g\in G.
 \end{equation*}
 Let $\ket{v}\in\C^n\setminus\{0\}$ be  
an eigenvector of $A$ and $\lambda\in\C$ be an eigenvalue to this eigenvector, 
i.e. $A\ket{v} = \lambda\ket{v}$. 
Then 
 \begin{equation*}
  A\urep{g}\ket{v} = \urep{g}A\ket{v} = \lambda\urep{g}\ket{v},
 \end{equation*}
 and thus $\urep{g}\ket{v}$ is as well an eigenvector to the eigenvalue 
 $\lambda$. This holds for all $\urep{g}$, and therefore the eigenspace 
 $S_\lambda = \left\{\ket{v}\in v: A\ket{v} = \lambda\ket{v} \right\}$ is 
 an invariant subspace under the unitary representation $\urep{g}$. By 
proposition \ref{thm:directsum}, the partition of $V$ into irreducible 
subspaces is unique, so $S_\lambda = \bigoplus_{i\in I_\lambda} V_i$, 
with $G$-submodules $V_i$ of $V$ and some (non-empty) index set $I_\lambda$. 
The matrix $A$ is diagonalizable, so $\bigoplus_{\lambda\in\sigma(A)} 
S_\lambda = V$. As previously shown in proposition \ref{thm:directsum}, we can 
choose the orthonormal base $\left\{\ket{u_i}:\ 1\leq i\leq n\right\}$ of $V$ 
such that it contains orthonormal bases for all irreducible $G$-submodules 
$V_i$. As we have shown those base vectors are all eigenvectors of $A$, and 
therefore $V$ has an orthonormal base of eigenvectors of $A$, and thus $A$ is 
unitarily diagonalizable:
\begin{equation*}
 A = \summe{i}{1}{n} \tilde{\lambda}_i \ket{u_i}\bra{u_i} = \summe{j}{1}{m}.
\lambda_j \pi_j.
\end{equation*}
Note that $\lambda_i$ (and obviously $\tilde{\lambda}_i$) are not neccessarily 
distinct.

\end{proof}

\begin{cor}(Schur's Lemma for irreducible unitary representations)\\
 Let $\urep{\cdot}$ be an irreducible unitary 
 representation of the group $G$ on the complex vector space $V$. If a matrix 
$A$ commutes with  $\urep{g}$ 
 for all $g\in G$, then A is a scalar matrix, i.e. 
 \begin{equation}
  A = \lambda \ind_n,\ \lambda\in\C.
 \end{equation}
\end{cor}
\begin{proof}
 The matrix $A$ has at least one eigenvector $\ket{v}\in V\setminus\{0\}$ with 
eigenvalue $\lambda\in\C$. We use the same argument as in lemma 
\ref{lemma:schur}, and conclude that the eigenspace $S_\lambda$ is an invariant 
subspace of $V$ with respect to $\urep{g}$. We know that $v\in S_\lambda$, 
and so by definition \ref{def:irred}, it must be $S_\lambda = V$. Now $\lambda$ 
is the only eigenvalue of $A$, which means that $A=\lambda\ind_n$.
\end{proof}

\section{The Haar Integral}

In this section we are going to establish a probability measure on the unitary 
group. As seen in section \ref{sec:ugroup}, there is a relation of the unitary 
group of dimension $n$ to the unit sphere in $\R^{2n}$. A probability measure 
on this sphere could be considered fair, or evenly distributed, if a change of  
orientation of a set does not change its probability. This would be 
the case if we take the probability proportional to the spheric area of the 
set. As it turns out, the Haar measure is equivalent to this notion of a 
probability distribution on the sphere (see \cite{christensen}).


% \begin{definition}\label{def:integral} (positive integral)\\
%  An integral $\mu$ on a group $G$ is a linear functional from the space 
%  \begin{equation*}
%  \Cont_+(G) = \Set{f: G\to\R| f(g)\geq 0, g\text{ continuous}} 
%  \end{equation*}
%  to the non-negative real numbers. The 
% integral is called \emph{positive}, if for all $f$ with compact support 
% $\mu(f)>0$. 
% \end{definition}
% 
% \begin{notation} (integrals and measures)\\
%   If $\mu$ is an integral on the group $G$, we write
%   \begin{equation}
%    \mu(f) \eqqcolon \int_G f(x)\Id\mu(x)
%   \end{equation}
%   for a continuous non-negative function $f$ like in definition 
%   \ref{def:integral}. 
%   
%   Sometimes we also want to refer to \emph{volumes} of subsets of $G$. We abuse 
% the notation and define the measure $\mu$ associated with the integral of the 
% same name as 
%   \begin{equation}
%    \mu(A) \coloneqq \int\limits_G \chi_A(x)\Id\mu(x)
%   \end{equation}
%   for a borel set $A\subseteq G$ and the \emph{characteristic function} $\chi$.
% \end{notation}

\begin{notation}
 We will sometimes use the terms measure and integral exchangeable. This is not 
a problem though, because we can always construct an integral from a given 
measure, and we get the measure from the integral by setting
 \begin{equation}
  \mu(A) \coloneqq \int\limits_G \chi_A(x)\Id\mu(x)
 \end{equation}
  for a Borel set $A\subseteq G$ and the \emph{characteristic function} $\chi$.
 
 We also want to introduce a short-hand notation for the translation of sets:
 \begin{equation}
  gA = \{gh\in G:\ h\in A\},\quad g\in G,\ A \subseteq G \text{ Borel set}.
 \end{equation}
\end{notation}


\begin{definition}(positive integral)\\
An integral $\mu$ is said to be positive, if it is positive for all 
continuous, $\mu$-integrable functions $f$ with compact support.
\end{definition}


\begin{definition} (left- and right-invariant integrals)\\
 An integral $\mu$ on the group $G$ is left-invariant, if 
 \begin{equation}
  \int_G f(gx)\Id\mu(x) = \int_G f(x)\Id\mu(x)
 \end{equation}
 for all $g\in G$ and all $\mu$-integrable functions $f$.
\end{definition}

We will not prove the following theorem, but we will gladly accept its 
results.
\begin{thm} (Alfr\'ed Haar, John von Neumann 1933; see \cite{haar})\\
 On every locally compact group $G$ exists at least one left-invariant positive 
integral. If $\mu,\nu$ are both left-invariant positive integrals on $G$, then 
there is a constant factor $c>0$ such that $\nu = c\mu$.
\end{thm}

\begin{cor}
Let $G$ be a compact group.
\begin{enumerate}\renewcommand{\labelenumi}{(\alph{enumi})}
 \item Every left-invariant Haar measure on $G$ is also a right-invariant 
Haar-Measure on $G$. 
 \item The Haar-measure on $G$ is finite.
 \end{enumerate}
\end{cor}
\begin{proof}
 Let $\mu$ be a left-invariant measure on $G$ and define the 
measure $\nu(A)\coloneqq \mu(gA)$ for Borel sets $A$ and some fixed $g\in G$. 
Then $\nu$ is \todo
\end{proof}



\section{Conclusion and Outlook}
We will conclude this report with an application of Schur's Lemma on random 
quantum states, as promised. 

\begin{example}(expectation value of random pure states)\footnote{As you may 
have noticed, we did not formally define an 
integral as a map from functions to matrices. We assume at this point that we 
can use the standard definition of matrix functions, acting on the 
eigenvalues.}\\
 We can model a \emph{random pure quantum state} $\rho = \ket{\psi}\bra{\psi}$ 
as some initial state $\rho_0 = \ket{\psi_0}\bra{\psi_0}$ that is 
transformed by a (Haar-)random unitary matrix $U$. We get 
\begin{equation}
 \E(\rho) = \int \ket{\psi}\bra{\psi}\Id\ket{\psi} = 
 \int_{\ugroup{n}} U\ket{\psi_0}\bra{\psi_0}U^\dagger\Id\mu(U).
\end{equation}
Note that this integral does not depend on the initial state $\rho_0$ because 
of the left-invariance of the Haar integral.

We see that $\E(\rho)$ is self-adjoint and thus (unitarily) diagonalizable. 
Furthermore, for any unitary matrix $Q$, we get 
\begin{align*}
 Q\E(\rho) &= \int_\ugroup{n} QU\rho U^\dagger\Id\mu(U)\\
 &= \int_\ugroup{n} W(U)\rho W(U)^\dagger Q\Id\mu(U)\\
 &= \int_\ugroup{n} W\rho W^\dagger Q\Id\mu(W) = \E(\rho)Q
\end{align*}
by substitution $W(U) = QU$ and the left-invariance of the Haar integral. 
Therefore, all prerequisites for lemma \ref{lemma:schur} are met, and we can 
write 
\begin{equation}
 \E(\rho) = \lambda\ind_n
\end{equation}
for some $\lambda\in\C$. The trace operator is linear, so we can exchange the 
integral and the trace operator and get 
\begin{equation}
 n\lambda = \trace(\E(\rho)) = \int_\ugroup{n} 
U\trace(\ket{\psi_0}\bra{\psi_0})U^\dagger\Id\mu(U) = \int_\ugroup{n}\Id\mu(U) 
= 1
\end{equation}
and this gives us the final result for the expectation value of a random pure 
state,
\begin{equation}
 \E(\rho) = \frac 1 n \ind_n,
\end{equation}
the maximally mixed state.
\end{example}

\todo Outlook and other results
% THATS IT
%%%%%%%%%%%%%%%%%%%%%%%%%%%%%%%%%%%%%%%%%%%%%%%%%%%%%%%%%%%%%%%%%%%%%%%%%%%%%%%%

%\cleardoublepage
% \clearpage

\begin{spacing}{1}
\bibliography{lit}{}
\bibliographystyle{alphadin}
% \clearpage
% \listoffigures
\end{spacing}

\end{document}


