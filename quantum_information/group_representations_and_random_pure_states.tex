\documentclass[a4paper,11pt, BCOR=4mm, DIV=12, pagesize]{scrartcl}


\thispagestyle{empty}
\pagestyle{empty}
\usepackage[utf8]{inputenc}
%\usepackage{ngerman}
%\usepackage[ngerman]{babel}
\usepackage[babel,german=quotes]{csquotes}
%\usepackage{exscale}
%\usepackage{dsfont}
%\usepackage{latexsym}
%\usepackage{theorem}   % change theoremstyle
%\usepackage[dvips]{color}


\usepackage{graphicx}
\usepackage{subfigure} 
\usepackage[singlelinecheck=true]{caption} % working out of ~
\usepackage{color}
% \usepackage{booktabs}
\usepackage{enumerate}
\usepackage{textcomp}
\usepackage{xparse}



\usepackage{amsmath}
\allowdisplaybreaks[1]
\usepackage{colonequals}
\usepackage{mathtools}
\usepackage{amsthm}
\usepackage{amssymb}
\usepackage{amsfonts}
\usepackage{bbm}
\usepackage{braket}


\usepackage[plainpages=false]{hyperref}
\definecolor{darkblue}{rgb}{0,0,.5}
\hypersetup{colorlinks=true, breaklinks=true, linkcolor=darkblue, 
menucolor=darkblue, urlcolor=darkblue}

\usepackage{setspace}


%==============================================================================
% theorem stuff
%\theoremstyle{break}  % add line break after theorem header
\newtheorem{thm}{Theorem}[section]
\newtheorem{cor}[thm]{Corollary}
\theoremstyle{definition}
\newtheorem{definition}[thm]{Definition}

% \newtheorem{satz}{Satz}[section]
% \newtheorem{lemma}[satz]{Lemma}
% \newtheorem{korollar}[satz]{Korollar}
% 
% \newtheorem{bsp}[satz]{Beispiel}
% \newtheorem{bem}[satz]{Bemerkung}
% \newtheorem{folgerung}[satz]{Folgerung}

% \newcommand*\defnpenalties{%
%   % Geklaut aus einem Posting von Michael Downes:
%   % > Set \clubpenalty, etc. to distinctive values for use
%   % > in tracing page breaks. These values are chosen so that
%   % > no single penalty will absolutely prohibit a page break,
%   % > but certain combinations of two or more will.
%   %
%   \clubpenalty=9996 \widowpenalty=9999 \brokenpenalty=4991
%   %
%   % > Reiterate the default value of \predisplaypenalty, for
%   % > completeness.
%   % >
%   % > Set postdisplaypenalty to a fairly high value to discourage
%   % > a page break between a display and a widow line at the end
%   % > of a paragraph.
%   %
%   \predisplaypenalty=10000 \postdisplaypenalty=1549
%   %
%   % > And then \displaywidowpenalty should be at least as high as
%   % > \postdisplaypenalty, otherwise in a situation where two
%   % > displays are separated by two lines, TeX will prefer to
%   % > break between the two lines, rather than before the first
%   % > line.
%   %
%   \displaywidowpenalty=1602
% }%
% 
% \let\sdef\definition
% \renewcommand{\def}{\defnpenalties\sdef}
% \let\sthm\thm
% \renewcommand{\thm}{\defnpenalties\sthm}
% \let\scor\cor
% \renewcommand{\cor}{\defnpenalties\scor}

%==============================================================================
% counter stuff
\renewcommand{\thefigure}{\thesection.\arabic{figure}}
\renewcommand{\theequation}{\thesection.\arabic{equation}}
\renewcommand{\thetable}{\thesection.\arabic{table}}

\newcommand{\resetsectioncounters}{%
\setcounter{figure}{0}%
\setcounter{table}{0}%
\setcounter{equation}{0}%
}




%==============================================================================
% my commands

\newcommand{\R}{\mathbb{R}} % real numbers
\newcommand{\N}{\mathbb{N}} % natural numbers
\newcommand{\C}{\mathbb{C}} % complex numbers
\renewcommand{\P}{\mathbb{P}}
\newcommand{\ind}{\mathbbm{1}}
\newcommand{\E}{\mathbb{E}}
\NewDocumentCommand\cmat{mg}{%
    \ensuremath{\C^{#1\times \IfNoValueTF{#2}{#1}{#2}}}%
}
\newcommand{\ugroup}[1]{U(#1)}
\newcommand{\urep}[1]{U_{#1}}


\newcommand{\needcite}{\textcolor{red}{[citation needed]}}
\newcommand{\needref}{\textcolor{red}{[local reference needed]}}
\newcommand{\needproof}{\textcolor{red}{[proof needed]}}
\newcommand{\todo}{\textcolor{red}{\Large[TODO]}}

\newcommand{\eps}{\varepsilon}
\renewcommand{\phi}{\varphi}
\newcommand{\Id}{\ \mathrm{d}}

\newcommand{\figref}[1]{(\ref{#1}, p. \pageref{#1})}


%==============================================================================
% my environments

\newenvironment{beweis}{\begin{proof}[Beweis]\defnpenalties}{\end{proof}}

%==============================================================================
%global settings
%\pagestyle{headings}

%\usepackage{a4wide}
%\usepackage{fullpage}
\pagestyle{plain}


\setcounter{tocdepth}{2}

\hyphenation{Zu-falls-stich-pro-be Zu-falls-stich-pro-ben Kon-fi-denz-wert Kon-fi-denz-wer-ten Kon-fi-denz-in-ter-vall}


%==============================================================================
% my settings






\onehalfspacing
\KOMAoptions{DIV=last}

\begin{document}
    \title{Random Unitary Crap}
\date{November 12, 2013}
\author{Markus D\"oring\\University of Heidelberg}
\maketitle
%\cleardoublepage
% \clearpage
% \pagenumbering{Roman}

	%\addcontentsline{toc}{chapter}{Inhalt}
% \begin{spacing}{1}
%     \tableofcontents
% 	\addcontentsline{toc}{section}{Inhaltsverzeichnis}
% 	\cleardoublepage  
%    \listoffigures
% 	\addcontentsline{toc}{section}{Abbildungsverzeichnis}
% 	\cleardoublepage  
% %     \listoftables
% % 	\addcontentsline{toc}{section}{Tabellenverzeichnis}
% % 	\cleardoublepage  
% \end{spacing}


\pagenumbering{arabic}

%%%%%%%%%%%%%%%%%%%%%%%%%%%%%%%%%%%%%%%%%%%%%%%%%%%%%%%%%%%%%%%%%%%%%%%%%%%%%%%%
% ACTUAL SEMINAR STUFF
\section{Preface}

This is the written report corresponding to my talk in the seminar 
\emph{Selected topics in Mathematical Physics: Quantum information theory}. The 
seminar was lead by Prof. Dr. Manfred Salmhofer, executive director of the 
Institute for Theoretical Physics at the University of Heidelberg, and Dr. 
Markus M\"uller, junior research group leader at that institute. 

The talk starts with a review of some properties of pure quantum states, 
unitary matrices and density matrices, followed by the introduction of group 
representations and the Haar measure. The final result will be the proof of 
Schur's Lemma and its use for calculating expectation values of random pure 
states.

\section{Motivation}

In fields like statistical dynamics or chaos theory, we use probabilistic 
models to come to conclusions about the state and the evolution of a system. 
If we want to apply quantum theory to these fields, it would be helpful to 
know whether a state whose classical analog is chaotic is effectively random 
\cite{wootters}. 

For the definition of a \emph{random quantum state} we need at least a 
probability measure on the set of quantum states. It turns out that the 
\emph{Haar measure} can be seen as a probability measure, and that 
some results of group representation theory allow us to calculate some results 
for random quantum states. 

In this seminar talk, we will start with some basics on the unitary group, then 
we will introduce the Haar measure as a probability measure on the unitary 
group, and we will close with the calculation of expectation values of random 
pure states:
\begin{equation}
\E(\ket{\psi}) = \int_{\ugroup{n}} \ket{\psi}\Id |\psi\rangle.
\end{equation}

\section{The Unitary Group}
Basics...

\section{The Haar Integral}

\begin{definition}\label{def:integral} (positive integral)\\
 An integral $\mu$ is a linear functional from the space 
 \begin{equation*}
 \Cont_+(G) = \Set{f: G\to\R| f(g)\geq 0, g\text{ continuous}} 
 \end{equation*}
 to the non-negative real numbers. The 
integral is called \emph{positive}, if for all $f$ with compact support 
$\mu(f)>0$.
\end{definition}

\begin{notation} (integral notation)\\
  If $\mu$ is an integral, we write
  \begin{equation}
   \mu(f) =: \int\limits_G f(x)\Id\mu(x)
  \end{equation}
  for a continuous non-negative function $f$ like in definition 
  \ref{def:integral}. 
  
  Sometimes we also want to refer to \emph{volumes} of subsets of $G$. We abuse 
the notation and define the measure $\mu$ associated with this integral as \todo
  \begin{equation}
   \mu(A) := \int\limits_G \Xi(x)\Id\mu(x)
  \end{equation}
  for a borel set $A\subseteq G$.
\end{notation}

\begin{thm} (Haar, see \cite{haar})
 On every locally compact group $G$ exists at least one 
\end{thm}



\section{Unitary Group Representations}
\begin{thm}(Representation as Direct Sum of invariant 
subspaces)\label{thm:directsum}\\
 bla
\end{thm}

\section{Schur's Lemma}

\begin{thm}(Schur's Lemma for unitary representations)\\
 Let
 \begin{align*}
  r: G&\to \ugroup{n}\\
  g&\mapsto \urep{g}
 \end{align*}
 be an irreducible unitary 
 representation of the group $G$. If $A\in\cmat{n}$ commutes with  $\urep{g}$ 
 for all $g\in G$, then A is a scalar matrix, i.e. 
 \begin{equation*}
  A = \lambda \ind_n,\ \lambda\in\C.
 \end{equation*}
\end{thm}
\begin{proof}
 We can state the commutativity of A as 
 \begin{equation*}
  A\urep{g} = \urep{g}A\quad \forall g\in G.
 \end{equation*}
 Let $\lambda\in\C$ be an eigenvalue of A and $\ket{v}\in\C^n\setminus\Set{ 0 
}$ be  an eigenvector to this  eigenvalue, i.e. $A\ket{v} = \lambda\ket{v}$. 
Then 
 \begin{equation*}
  A\urep{g}\ket{v} = \urep{g}A\ket{v} = \lambda\urep{g}\ket{v},
 \end{equation*}
 and thus $\urep{g}\ket{v}$ is as well an eigenvector to the eigenvalue 
 $\lambda$. This holds for all $\urep{g}$, and therefore the eigenspace 
 $S_\lambda = \left\{\ket{v}\in\C^n: A\ket{v} = \lambda\ket{v} \right\}$ is 
 an invariant subspace under the unitary representation $r$. The representation 
 is irreducible, so by theorem \ref{thm:directsum} either $S_\lambda = \{0\}$ 
 or $S_\lambda = \C^n$ holds. Since $0\neq\ket{v}\in S_\lambda$, we get 
$S_\lambda  = \C^n$. This means that $\lambda$ is the only eigenvalue of $A$, 
and we can  write for some orthonormal base $\left\{\ket{u_i}, 1\leq i\leq 
n\right\}$ of $\C^n$:
 \begin{equation*}
  A = \sum\limits_{i=1}^n \lambda \ket{u_i} \bra{u_i} = \lambda\ind_n
 \end{equation*}






\end{proof}



\section{Conclusion and Outlook}
We have shown \todo. Next talk will be about \todo.

\newpage
\section{BBB}
\cite{rep}
\cite{haar}
\begin{figure}[ht]
 \caption[short random description]{blub} 
 \label{fig:42}
\end{figure}



% THATS IT
%%%%%%%%%%%%%%%%%%%%%%%%%%%%%%%%%%%%%%%%%%%%%%%%%%%%%%%%%%%%%%%%%%%%%%%%%%%%%%%%

%\cleardoublepage
% \clearpage

\begin{spacing}{1}
\bibliography{lit}{}
\bibliographystyle{alphadin}
% \clearpage
\listoffigures
\end{spacing}

\end{document}


